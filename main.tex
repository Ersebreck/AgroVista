\documentclass[12pt]{article}
\usepackage[margin=1in]{geometry}
\usepackage{titlesec}
\usepackage{hyperref}
\usepackage{enumitem}
\usepackage{lmodern}
\usepackage{graphicx}
\usepackage{fancyhdr}
\usepackage{float}
\pagestyle{fancy}
\fancyhf{}
\rhead{MVP Agricultural Land Project}
\lhead{Erick Lozano Roa}
\rfoot{\thepage}

\titleformat{\section}{\normalfont\Large\bfseries}{\thesection}{1em}{}
\titleformat{\subsection}{\normalfont\large\bfseries}{\thesubsection}{1em}{}

\title{AgroVista \\ Agricultural Land Visualization Software}
\author{Erick Lozano Roa}
\date{\today}

\begin{document}

\maketitle

\section{Part 1 – Project Objective}

\subsection{Main Objective:} Allow agricultural landowners to visualize and manage, in a simple and centralized way, the information of their large properties and subdivisions (plots), including general and specific data for each.

\subsection{Problem to Solve:} Currently, there is no clear, accessible, and visual way to monitor or access key information about agricultural plots and lands, making decision-making and operational control difficult.

\section{Part 2 – User Stories}

\subsection*{Role: Owner}
\begin{itemize}[label=--]
  \item As an owner, I want to see a visual map with all my large lands and their subdivisions to get a quick overview of the state of my properties.
  \item As an owner, I want to access a summary of each plot to make strategic decisions about its use.
\end{itemize}

\subsection*{Role: User (Manager)}
\begin{itemize}[label=--]
  \item As a user, I want to select a plot and record completed tasks to maintain an up-to-date history.
  \item As a user, I want to visualize specific information for each plot to perform tasks correctly.
\end{itemize}

\subsection*{Role: Administrator}
\begin{itemize}[label=--]
  \item As an administrator, I want to create, edit, and delete lands and plots to keep the system up to date.
  \item As an administrator, I want to manage each user's permissions to control access to information.
\end{itemize}

\section{Part 3 – Data Modeling}

\subsection*{Entity: User}
Fields: id, name, email, password, role.

\subsection*{Entity: Land}
Fields: id, name, description, owner\_id, location\_id.

\subsection*{Entity: Plot}
Fields: id, name, current\_use, status, land\_id, location\_id.

\subsection*{Entity: Activity}
Fields: id, type, date, description, user\_id, plot\_id.

\subsection*{Entity: Geospatial Location}
Fields: id, type (point or polygon), coordinates (GeoJSON), reference (optional).

\section{Part 4 – Minimum Viable Product (MVP)}

\textbf{Includes:}
\begin{itemize}
  \item Visualization of lands and plots on a map.
  \item Detailed view of plots.
  \item Activity logging.
  \item Basic data creation and editing (admin).
  \item Report generation (optional).
  \item Support for recursive subdivisions at multiple levels.
\end{itemize}

\section{Part 5 – Wireframes}

\textbf{Key Guidelines:}
\begin{itemize}
  \item Focus the design around a central map.
  \item Split the interface into map + side panel.
  \item Design progressive views (general → detailed).
  \item Use icons or colors to differentiate states.
\end{itemize}

\subsection{Wireframes}

Below is the first iteration:

\begin{figure}[H]
    \centering
    \includegraphics[width=0.5\textwidth]{images/p_principal.png}
    \caption{Main screen}
    \label{fig:enter-label}
\end{figure}

On clicking the blue bar:

\begin{figure}[H]
    \centering
    \includegraphics[width=0.5\textwidth]{images/p_azul.png}
    \caption{Slider bar}
    \label{fig:enter-label}
\end{figure}
\vspace{1 in}
On clicking the map:

\begin{figure}[H]
    \centering
    \includegraphics[width=0.5\textwidth]{images/p_mapa_g_1.png}
    \caption{Screen with general map}
    \label{fig:enter-label}
\end{figure}

On clicking the red map:

\begin{figure}[H]
    \centering
    \includegraphics[width=0.5\textwidth]{images/p_mapa_g_2.png}
    \caption{Screen with divided map}
    \label{fig:enter-label}
\end{figure}

On clicking a sub-land:

\begin{figure}[H]
    \centering
    \includegraphics[width=0.5\textwidth]{images/p_mapa_p.png}
    \caption{Screen with plot map}
    \label{fig:enter-label}
\end{figure}

\vspace{1 in}

\section{Part 6 – Future Scope of the Project}

\begin{itemize}
  \item Short-term project for practice, but with real potential for use.
  \item Potential real clients to validate the MVP.
  \item Scalable to a SaaS system for land management.
\end{itemize}

\section{Part 7 – System Components}

\subsection*{Web Frontend}
\textbf{Tool:} Streamlit.\\
\textbf{Functionality:} Interactive map and information cards visualization.

\subsection*{Backend}
\textbf{Tool:} FastAPI.\\
\textbf{Functionality:} Expose data and basic system logic.

\subsection*{Database}
\textbf{Tool:} PostgreSQL + PostGIS.\\
\textbf{Functionality:} Storage for structured and spatial data.

\subsection*{Geospatial Visualization}
\textbf{Tool:} Folium or Pydeck.\\
\textbf{Functionality:} Interactive map on the user interface.

\section{Part 8 – Technology Stack}

\begin{itemize}
  \item \textbf{Frontend:} Streamlit + Folium.
  \item \textbf{Backend:} FastAPI (Python).
  \item \textbf{Database:} PostgreSQL + PostGIS.
  \item \textbf{ORM:} SQLAlchemy (optional).
  \item \textbf{Useful libraries:} geopandas, shapely, streamlit-folium.
  \item \textbf{Local deployment:} Docker.
\end{itemize}

\section{Part 9 – ???}
\begin{itemize}
  \item PERMACULTURE AS A SERVICE
\end{itemize}

\begin{figure}[H]
    \centering
    \includegraphics[width=0.5\linewidth]{images/logo_agrovista.png}
    \label{fig:enter-label}
\end{figure}

\end{document}
